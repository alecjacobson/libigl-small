\documentclass{diary}
\title{Active set solver for quadratic programming}
\author{Alec Jacobson}
\date{18 September 2013}

\renewcommand{\A}{\mat{A}}
\renewcommand{\Q}{\mat{Q}}
\newcommand{\Aeq}{\mat{A}_\text{eq}}
\newcommand{\Aieq}{\mat{A}_\text{ieq}}
\newcommand{\Beq}{\vc{B}_\text{eq}}
\newcommand{\Bieq}{\vc{B}_\text{ieq}}
\newcommand*\Bell{\ensuremath{\boldsymbol\ell}}
\newcommand{\lx}{\Bell}
\newcommand{\ux}{\vc{u}}

\begin{document}
Quadratic programming problems (QPs) can be written in general as:
\begin{align}
\argmin \limits_\Z &
  \Z^\transpose \A \Z + \Z^\transpose \B + \text{ constant}\\
\text{subject to } & \Aieq \Z ≤ \Bieq,
\end{align}
where $\Z \in \R^n$ is a vector of unknowns,  $\A \in \R^{n \times n}$ is a (in
our case sparse) matrix of quadratic coefficients, $\B \in \R^n$ is a vector of
linear coefficients, $\Aieq \in \R^{m_\text{ieq} \times n}$ is a matrix (also
sparse) linear inequality coefficients and $\Bieq \in \R^{m_\text{ieq}}$ is a
vector of corresponding right-hand sides. Each row in $\Aieq \Z ≤ \Bieq$
corresponds to a single linear inequality constraint.

Though representable by the linear inequality constraints above---linear
\emph{equality} constraints, constant bounds, and constant fixed values appear
so often that we can write a more practical form
\begin{align}
\argmin \limits_\Z &
  \Z^\transpose \A \Z + \Z^\transpose \B + \text{ constant}\\
\text{subject to } & \Z_\text{known} = \Y,\\
                   & \Aeq \Z = \Beq,\\
                   & \Aieq \Z ≤ \Bieq,\\
                   & \Z ≥ \lx,\\
                   & \Z ≤ \ux,
\end{align}
where $\Z_\text{known} \in \R^{n_\text{known}}$ is a subvector of our unknowns $\Z$ which
are known or fixed to obtain corresponding values $\Y \in \R^{n_\text{known}}$,
$\Aeq \in \R^{m_\text{eq} \times n}$ and $\Beq \in \R^{m_\text{eq} \times n}$
are linear \emph{equality} coefficients and right-hand sides respectively, and
$\lx, \ux \in \R^n$ are vectors of constant lower and upper bound constraints.

\todo{This notation is unfortunate. Too many bold A's and too many capitals.}

This description exactly matches the prototype used by the
\texttt{igl::active\_set()} function.

The active set method works by iteratively treating a subset (some rows) of the
inequality constraints as equality constraints. These are called the ``active
set'' of constraints. So at any given iterations $i$ we might have a new
problem:
\begin{align}
\argmin \limits_\Z &
  \Z^\transpose \A \Z + \Z^\transpose \B + \text{ constant}\\
\text{subject to } & \Z_\text{known}^i = \Y^i,\\
                   & \Aeq^i \Z = \Beq^i,
\end{align}
where the active rows from 
$\lx ≤ \Z ≤ \ux$ and $\Aieq \Z ≤ \Bieq$  have been appended into
$\Z_\text{known}^i = \Y^i$ and $\Aeq^i \Z = \Beq^i$ respectively.

This may be optimized by solving a sparse linear system, resulting in the
current solution $\Z^i$. For equality constraint we can also find a
corresponding Lagrange multiplier value. The active set method works by adding
to the active set all linear inequality constraints which are violated by the
previous solution $\Z^{i-1}$ before this solve and then after the solve
removing from the active set any constraints with negative Lagrange multiplier
values.

%\begin{pullout}
%while not converged
%  add to active set all rows where $\Aieq \Z > \Bieq$, $\Z < \lx$ or $\Z > \ux$
%  solve problem treating active constraints as equality constraints
%  remove from active set all rwos 
%end
%\end{pullout}

The fixed values constraints of $\Z_\text{known}^i = \Y^i$ may be obtained by
substituting $\Z_\text{known}^i$ for $\Y^i$ in the energy directly.
Corresponding Lagrange multiplier values $\lambda_\text{known}^i$ can be
recovered after the fact.

% HURRY UP AND GET TO THE QR DECOMPOSITION

\end{document}
