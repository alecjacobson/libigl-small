\documentclass[12pt]{diary}
\immediate\write18{bibtex \jobname}

\title{Papers implemented in \textsc{libigl}}
\author{Alec Jacobson}
\date{last revised 22 April 2014}

\begin{document}
This document serves as a companion reference to better list the references to
scientific articles implemented within \textsc{libigl}. It will no doubt be
incomplete.

\paragraph{\texttt{cotmatrix}, \texttt{massmatrix}}
build discrete operators on triangle and tetrahedral meshes. 
\cite{Pinkall:1993:CDM,meyer03ddo,Jacobson:THESIS:2013}. For tet meshes, we no
longer use the ``by the book'' FEM construction \`a la \cite{Sharf:2007vv},
rather a purely geometric approach \cite{Barth:1994,Xu:1999}.

\paragraph{\texttt{harmonic}} solves a Laplace problem (equivalently
minimizes the Dirichlet energy) with some simple boundary conditions
\cite{HarmonicCoodinates07}. There's also an option to solve
``higher order Laplace problems'' (bi-Laplace, tri-Laplace, etc.)
\cite{Botsch:2004:AIF,sorkine04lsm,Jacobson:MixedFEM:2010}.

\paragraph{\texttt{bbw/}} implements ``bounded biharmonic
weights'' \cite{Jacobson:BBW:2011}.

\paragraph{\texttt{svd3x3/arap}} is a generalized implementation
for solving ``as-rigid-as-possible'' (ARAP) mesh deformation or parameterization
problems \cite{ARAP_modeling:2007,Liu:2008:ALA,Chao:2010:ASG}.

\paragraph{\texttt{svd3x3/arap\_dof}} implements ``FAST'',
which is simultaneously a reduced form of ARAP and a method for automatically
choosing skinning transformations \cite{Jacobson:FAST:2012}.

\paragraph{\texttt{dqs}} implements ``Dual quaternion skinning''
\cite{Kavan:2008:GSW}.

\paragraph{\texttt{lbs}} implements ``linear blend skinning'', also known as
``skeletal subspace deformation'', or ``enveloping''. This technique is often
attributed to \cite{Magnenat-Thalmann:1988:JLD}.

\bibliographystyle{acmsiggraph}
\bibliography{references} 

\end{document}
__END__
